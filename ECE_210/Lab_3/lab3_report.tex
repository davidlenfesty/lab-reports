\documentclass{article}

\usepackage{graphicx}
\usepackage{setspace}
\usepackage{listings}
\usepackage{color}
\usepackage{circuitikz}
\usepackage{float}


\title{ECE 210 - Combinational Logic Design \\ Lab 3}
\date{2018-11-07}
\author{Radomir Wasowski \\ wasowski@ualberta.ca
        \and David Lenfesty \\ lenfesty@ualberta.ca}

\setcounter{tocdepth}{2} % Show subsections

\definecolor{dkgreen}{rgb}{0,0.6,0}
\definecolor{gray}{rgb}{0.5,0.5,0.5}
\definecolor{mauve}{rgb}{0.58,0,0.82}

\lstset{basicstyle=\small,
        keywordstyle=\color{mauve},
        identifierstyle=\color{dkgreen},
        stringstyle=\color{gray},
        numbers=left
        }

\begin{document}

\pagenumbering{gobble}
\doublespacing
\maketitle
\newpage

\singlespacing

\section{Abstract}

Display technology allows designers to display changing and important information to users.
By using logic gates to implement a 7-segment display that can directly output the value of a sensor,
an effective user interface can be built.

\section{Introduction}



\section{Design}

\paragraph{Part 1:}

Because many characters, when displayed on a 7-segment display, share common elements,
it is relatively trivial to combine logic signals using minterms or maxterms.
Using Karnaugh maps of each different letter based on the 4-bit input, a series of logical outputs
were found, which were then connected to the appropriate display segments.
Because of this, the number of discrete logic elements could be vastly reduced.
Once these signals were found, they were transcribed into a VHDL file,
to be uploaded to an FPGA for testing.

The resulting code can be found below:


\section{Results}


\section{Discussion}


\section{Conclusion}

\end{document}
