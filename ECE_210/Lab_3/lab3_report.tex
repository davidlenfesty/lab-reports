\documentclass{article}

\usepackage{graphicx}
\usepackage{setspace}
\usepackage{listings}
\usepackage{color}
\usepackage{circuitikz}
\usepackage{float}


\title{ECE 210 - Combinational Logic Design \\ Lab 3}
\date{2018-11-07}
\author{Radomir Wasowski \\ wasowski@ualberta.ca
        \and David Lenfesty \\ lenfesty@ualberta.ca}

\setcounter{tocdepth}{2} % Show subsections

\definecolor{dkgreen}{rgb}{0,0.6,0}
\definecolor{gray}{rgb}{0.5,0.5,0.5}
\definecolor{mauve}{rgb}{0.58,0,0.82}

\lstset{basicstyle=\small,
        keywordstyle=\color{mauve},
        identifierstyle=\color{dkgreen},
        stringstyle=\color{gray},
        numbers=left
        }

\begin{document}

\pagenumbering{gobble}
\doublespacing
\maketitle
\newpage

\singlespacing

\section{Abstract}

Display technology allows designers to display changing and important information to users.
By using logic gates to implement a 7-segment display that can directly output the value of a sensor,
an effective user interface can be built.

\section{Introduction}



\section{Design}

\paragraph{Part 1:}

Because many characters, when displayed on a 7-segment display, share common elements,
it is relatively trivial to combine logic signals using minterms or maxterms.
Using Karnaugh maps of each different letter based on the 4-bit input, a series of logical outputs
were found, which were then connected to the appropriate display segments.
Because of this, the number of discrete logic elements could be vastly reduced.
Once these signals were found, they were transcribed into a VHDL file,
to be uploaded to an FPGA for testing.

\paragraph{Part 2:}

Instead of defining all of the logic gates manually, VHDL also supports
behavioural programming.
This style is different in that the synthesizer will decide how to implement
the code in logic, instead of the programmer doing the heavy lifting.
By utilizing a case statement, writing the program could be made easier,
as the designer does not have to worry about exactly how the logic should work.
However, this method does take some of the control over how the circuit operates
away from the designer.

\paragraph{Part 3:}

By using one of the inbuilt clocks on the FPGA board, a counter can be used to implement an
upcounter with a delay.
The signal from the clock can be sent to an internal counter which can
be divided to fit a 1 second signal, which can then go to another counter, which can output directly to
the 7 segment display.

\paragraph{Part 4:}

If the signal controlling which of the two displays are on is inverted 
extremely quickly (>~200Hz), we can take advantage of the limitations
of the human eye to make it appear that both displays are on at the same time.
This functionality can easily be implement by using behavioural logic to
toggle the control signal on the positive edge of every clock cycle.

\section{Results}

For all parts of this lab, the display behaved as expected,
displaying the corrected hexadecimal digit for the 4-bit input.
Using the example code, the counter updated once a second, and we were able
to show this information on two screens at once.


    \begin{figure}[H]
        % Mess around with widths later
        \includegraphics[width=115mm]{display_single.jpg}
        \caption{Single-character hexadecimal deisplay.}
        \label{fig:display_single}
    \end{figure}

    
    \begin{figure}[H]
        % Mess around with widths later
        \includegraphics[width=115mm]{counter_dual.jpg}
        \caption{Dual character upcounter.}
        \label{fig:counter_dual}
    \end{figure}

\section{Discussion}


\section{Conclusion}

\end{document}
