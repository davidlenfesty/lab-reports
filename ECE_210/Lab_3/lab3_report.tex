\documentclass{article}

\usepackage{graphicx}
\usepackage{setspace}
\usepackage{listings}
\usepackage{color}
\usepackage{circuitikz}
\usepackage{float}


\title{ECE 210 - Combinational Logic Design \\ Lab 3}
\date{2018-11-07}
\author{Radomir Wasowski \\ wasowski@ualberta.ca
        \and David Lenfesty \\ lenfesty@ualberta.ca}

\setcounter{tocdepth}{2} % Show subsections

\definecolor{dkgreen}{rgb}{0,0.6,0}
\definecolor{gray}{rgb}{0.5,0.5,0.5}
\definecolor{mauve}{rgb}{0.58,0,0.82}

\lstset{basicstyle=\small,
        keywordstyle=\color{mauve},
        identifierstyle=\color{dkgreen},
        stringstyle=\color{gray},
        numbers=left
        }

\begin{document}

\pagenumbering{gobble}
\doublespacing
\maketitle
\newpage

\singlespacing

\section{Abstract}

Display technology allows designers to display changing and important information to users.
By using logic gates to implement a 7-segment display that can directly output the value of a sensor,
an effective user interface can be built.

\section{Introduction}



\section{Design}

\paragraph{Part 1:}

Because many characters, when displayed on a 7-segment display, share common elements,
it is relatively trivial to combine logic signals using minterms or maxterms.
Using Karnaugh maps of each different letter based on the 4-bit input, a series of logical outputs
were found, which were then connected to the appropriate display segments.
Because of this, the number of discrete logic elements could be vastly reduced.
Once these signals were found, they were transcribed into a VHDL file,
to be uploaded to an FPGA for testing.

The resulting code can be found below:

\textbf{Should we include the code? I feel it is very simple and could easily
be explained.}

\begin{lstlisting}[language=VHDL]
entity part1 is
    Port ( sw : in STD_LOGIC_VECTOR (3 downto 0);
           out_7seg : out STD_LOGIC_VECTOR (6 downto 0);
           CC : out STD_LOGIC);
end part1;

architecture Behavioral of part1 is
    signal I : STD_LOGIC_VECTOR (18 downto 1);
begin
     CC <= '0';

     I(1)  <= not sw(2) and not sw(0);
     I(2)  <=  not sw(3) and sw(1) and sw(0);
     I(3)  <= sw(3) and not sw(1) and not sw(0);
     I(4)  <= sw(3) and not sw(2) and not sw(1);
     I(5)  <= sw(2) and sw(1) and not sw(0);
     I(6)  <= sw(3) and sw(2) and sw(1);
     I(7)  <= not sw(3) and sw(2) and sw(0);
     I(8)  <= not sw(3) and not sw(1) and not sw(0);
     I(9)  <= sw(3) and not sw(1) and sw(0);
     I(10) <= not sw(2) and not sw(1);
     I(11) <= not sw(2) and sw(1) and sw(0);
     I(12) <= sw(3) and not sw(2) and sw(1);
     I(13) <= not sw(3) and sw(2);
     I(14) <= sw(2) and not sw(1) and sw(0);
     I(15) <= not sw(3) and not sw(2) and not sw(0);
     I(16) <= sw(3) and sw(2);
     I(17) <= not sw(3) and sw(2) and not sw(1);
     I(18) <= not sw(2) and sw(1);
     
     out_7seg(0) <= I(1) or I(2) or I(3)
        or I(4) or I(5) or I(6) or I(7);
     out_7seg(1) <= I(1) or I(2) or I(8)
        or I(9) or I(10);
     out_7seg(2) <= I(9) or I(10) or I(11)
        or I(12) or I(13);
     out_7seg(3) <= I(3) or I(4) or I(5)
        or I(11) or I(14) or I(15);
     out_7seg(4) <= I(1) or I(3) or I(5)
        or I(12) or I(16);
     out_7seg(5) <= I(3) or I(4) or I(5)
        or I(6) or I(8) or I(12) or I(17);
     out_7seg(6) <= I(4) or I(5) or I(6)
        or I(9) or I(17) or I(18);


end Behavioral;
\end{lstlisting}


\section{Results}


\section{Discussion}


\section{Conclusion}

\end{document}
